% Einführung
\section{Einführung}

Diese Arbeit beleuchtet verschiedene Musteranwendungen der Methode der Data-Envelopment-Analysis (DEA) zum Vergleich mehrerer Entscheidungseinheiten (Decission Making Units, DMUs).
Da es sich bei DEA um eine Methode mit sehr vielfältigen Anwendungsgebieten handelt, sind die hier vorgestellten Anwendungsfälle nur als beispielhaft zu betrachten.

Wenngleich jeder Anwendungsfall eigene Fragen aufwirft, so ist im Zusammenhang mit DEA meistens auf eine Reihe von Aspekten jedenfalls zu achten.
Da deren Auswahl letztendlich entscheidend für das Ergebnis der Untersuchung ist, kommt eine zentrale Rolle den zu analysierenden Input- und Output-Faktoren zu.
Es muss auch berücksichtigt werden, dass die relevanten Faktoren nicht nur vom Untersuchungsgegenstand (und der Datenlage) sondern insbesondere von der jeweiligen Fragestellung abhängen.
Für denselben Untersuchungsgegenstand können mit wechselnder Fragestellung daher unterschiedliche Input-/Output-Größen maßgeblich sein.
Weiters sind die verbreitetsten Methoden von DEA auf die Anwendung mit Cross-Sectional-Daten ausgelegt.
Daher wird in der Regel die Effizienz in einem bestimmten Zeitpunkt untersucht -- Veränderungen im Laufe der Zeit bleiben meist unberücksichtigt.
Zwar gibt es Möglichkeiten, mithilfe von Malmquist-Indexen Panel-Daten zu untersuchen, auch hier werden aber nur 2 Zeitperioden miteinander verglichen.

In \cite*[Abschnitt 3.10 "`Sample Applications"')]{fried_measurement_2008} werden fünf Beispielanwendungen von DEA vorgestellt, nämlich die Untersuchung von Schulen in England, von Filialen einer portugiesischen Bank, der Regulation privater englischer und walisischer Wasseranbieter, von Pubs einer englischen Brauerei, sowie englischer und walisischer Polizeibezirke.
Die präsentierten Anwendungsfälle basieren auf publizierten Studien.
In den einzelnen Teilabschnitten der Subkapitel des Buches, auf die im folgenden detailliert eingegangen wird, werden die verwendeten Input- und Output-Parameter präsentiert und die Zielsetzung und Konzeption der jeweiligen Studie erläutert.
Es wird auch auf die Besonderheiten der gegenständlichen Studie hingewiesen.