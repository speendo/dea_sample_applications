% Education
\section{Vergleich von Schulen in England}
\subsection{Einführung}
Die vergleichende Analyse von Bildungseinrichtungen ist ein beliebtes Anwendungsgebiet von DEA.
In verschiedenen Studien werden Schulen, höhere Bildungseinrichtungen, SchülerInnen oder Schulbezirke miteinander verglichen (Beispiele für die unterschiedlichen Bereiche werden in \cite*[S.77]{fried_measurement_2008} angeführt).

Die gegenständliche Studie wurde von der englischen local education authority (LEA) in Auftrag gegeben und in \cite{thanassoulis_guiding_1994} dokumentiert.
Sie vergleicht 14 englische Schulen nach den Ergebnissen von 16-jährigen SchülerInnen beim General Certificate of Secondary Education (GCSE) Examen das im englischen Schulsystem die mittlere Reife dokumentiert.

\subsection{Verwendete Parameter}
\subsubsection{Output-Parameter}
Die Outputparameter der Studie sollen die Ausbildungsqualität der jeweiligen Schule dokumentieren.
Genauer gesagt wird die Qualifikation der AbsolventInnen herangezogen.
Zu diesem Zweck werden zwei Parameter ausgewählt: das durchschnittliche Resultat der AbsolventInnen beim GCSE ("`Average GCSE score per pupil"') sowie der Anteil der SchülerInnen, die nach Abschluss des GCSE einer Erwerbsarbeit nachgehen ("`Percentage of pupils not unemployed after GCSEs"').
Die Autoren der Studie weisen darauf hin, dass es im Zusammenhang mit der zweiten Variable generell notwendig wäre, die ökonomischen Rahmenbedingungen im Einzugsgebiet der jeweiligen Schule zu berücksichtigen, da diese einen Einfluss auf die Beschäftigungsaussichten der AbsolventInnen haben.
Im konkreten Fall sind den Autoren zufolge die ökonomischen Unterschiede zwischen den Einzugsgebieten aber klein genug, um diesen Aspekt außer Acht zu lassen.

\subsubsection{Inputparameter}
Als Inputparameter wurden ebenfalls zwei Faktoren herangezogen.
Mit dem ersten Aspekt versucht man den Anteil der SchülerInnen, die \emph{nicht} sozial benachteiligt sind anzunähern.
Als Indikator für soziale Benachteiligung zieht man den Bezug von (gratis) Essensmarken für die Schulkantine heran. Der beobachtete Inputparameter ist somit der Prozentsatz der SchülerInnen, die keine Essensmarktn beziehen ("`Percentage of pupils not receiving free school meals"').
Als zweiter Input wird das durchschnittliche Ergebnis der SchülerInnen bei einem Test über Sprachintelligenz beim Schuleintritt verwendet ("`Mean verbal reasoning score per pupil on entry"').

Die Auswahl der Inputparameter resultiert aus Forschungsergebnissen, die bescheinigen, dass diese Faktoren gut geeignet sind, um das Ergebnis beim CGSE-Test zu prognostizieren.
Somit wird durch das Einbeziehen dieser Faktoren als Inputparameter ein Teil des Outcomes bereits erklärt.
Der durch die Inputfaktoren nicht erklärbare Anteil der Outputparameter sollte somit zu guten Teilen mit den Qualitäten bzw. der Effizienz der jeweiligen Bildungseinrichtung in Zusammenhang stehen.