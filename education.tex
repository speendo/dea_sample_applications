% Education
\section{Vergleich von Schulen in England}
\subsection{Einführung}
Die vergleichende Analyse von Bildungseinrichtungen ist ein beliebtes Anwendungsgebiet von DEA.
In verschiedenen Studien werden Schulen, höhere Bildungseinrichtungen, SchülerInnen oder Schulbezirke miteinander verglichen (Beispiele für die unterschiedlichen Bereiche werden in \cite*[S.77]{fried_measurement_2008} angeführt).

Die gegenständliche Studie wurde von der englischen local education authority (LEA) in Auftrag gegeben und in \cite{thanassoulis_guiding_1994} dokumentiert.
Sie vergleicht 14 englische Schulen nach den Ergebnissen von 16-jährigen SchülerInnen beim General Certificate of Secondary Education (GCSE) Examen das im englischen Schulsystem die mittlere Reife dokumentiert.

\subsection{Verwendete Parameter}
\subsubsection{Output-Parameter}
Die Outputparameter der Studie sollen die Ausbildungsqualität der jeweiligen Schule dokumentieren.
Genauer gesagt wird die Qualifikation der AbsolventInnen herangezogen.
Zu diesem Zweck werden zwei Parameter ausgewählt: das durchschnittliche Resultat der AbsolventInnen beim GCSE ("`Average GCSE score per pupil"') sowie der Anteil der SchülerInnen, die nach Abschluss des GCSE nicht arbeitslos sind ("`Percentage of pupils not unemployed after GCSEs"').
Die Autoren der Studie weisen darauf hin, dass es im Zusammenhang mit der zweiten Variable generell notwendig wäre, die ökonomischen Rahmenbedingungen im Einzugsgebiet der jeweiligen Schule zu berücksichtigen, da diese einen Einfluss auf die Beschäftigungsaussichten der AbsolventInnen haben.
Im konkreten Fall sind den Autoren zufolge die ökonomischen Unterschiede zwischen den Einzugsgebieten aber klein genug, um diesen Aspekt außer Acht zu lassen.

\subsubsection{Inputparameter}
Als Inputparameter wurden ebenfalls zwei Faktoren herangezogen.
Mit dem ersten Aspekt versucht man den Anteil der SchülerInnen, die \emph{nicht} sozial benachteiligt sind anzunähern.
Als Indikator für soziale Benachteiligung zieht man den Bezug von (gratis) Essensmarken für die Schulkantine heran. Der beobachtete Inputparameter ist somit der Prozentsatz der SchülerInnen, die keine Essensmarktn beziehen ("`Percentage of pupils not receiving free school meals"').
Als zweiter Input wird das durchschnittliche Ergebnis der SchülerInnen bei einem Test über Sprachintelligenz beim Schuleintritt verwendet ("`Mean verbal reasoning score per pupil on entry"').

Die Auswahl der Inputparameter resultiert aus Forschungsergebnissen, die bescheinigen, dass diese Faktoren gut geeignet sind, um das Ergebnis beim CGSE-Test zu prognostizieren.
Somit wird durch das Einbeziehen dieser Faktoren als Inputparameter ein Teil des Outcomes bereits erklärt.
Der durch die Inputfaktoren nicht erklärbare Anteil der Outputparameter sollte somit zu guten Teilen mit den Qualitäten bzw. der Effizienz der jeweiligen Bildungseinrichtung in Zusammenhang stehen.

% Bild von Input-/Outputfaktoren
\subsection{Resultate}
Eines der Ergebnisse der Studie ist zweifellos der Vergleich zwischen den einzelnen Schulen.
Dadurch steht der LEA ein Ranking zur Verfügung, das neben der im Ranking enthaltenen Information an sich, auch für Entscheidungen zur Schulentwicklung hilfreich ist.
Neben dem reinen Benchmark bietet \cite{thanassoulis_guiding_1994} aber auch eine Art "`Entwicklungspfad"' für ineffiziente Schulen: für jede Schule kann ein oder mehrere effiziente(r) "`peer(s)"' identifiziert werden.
Diese Peers können als Vorbild für eine Entwicklung Richtung Effizienz dienen.

Zum Beispiel hat Schule Sc9010 eine relative Effizienz von \SI{72,74}{\percent}.
Die effiziente Peer-Shule Sc912 hat ähnliche Verbal-Reasoning-Werte, aber um \SI{14}{\percent} höhere Werte beim Faktor "`Schülern ohne soziale Benachteiligung"'.
Die Outputs bei Schule Sc912 unterscheiden sich gegenüber Sc9010 aber noch drastischer nach oben: die GCSE-Ergebnisse sind durchschnittlich um \SI{54}{\percent} besser und fast doppelt so viele AbsolventInnen sind nach dem Schulabschluss nicht arbeitslos.
Um mit den Inputfaktoren von Sc9010 effizient zu werden, müsste der GCSE-Score dieser Schule auf 55,67 steigen und \SI{25}{\percent} (statt \SI{17,34}{\percent}) der AbsolventInnen dürften nicht arbeitslos sein.

% Tabelle 3.11

Diese Zielsetzung spiegelte das formale Konzept von "`radial improvement"' wider.
Allerdings kann auch einem Outcome-Parameter der Vorzug gegeben werden.
Würde etwa Schule Sc8914 beide Outcomes gleich bewerten, müssten beide Faktoren um \SI{24}{\percent} verbessert werden.
Wenn allerdings im GCSE-Ergebnis ein Schwerpunkt gesetzt werden soll, müsste dieses um \SI{42}{\percent}, und die andere Variable nur um \SI{17}{\percent} verbessert werden.
Wie sich zeigt, darf die Schule in diesem Fall keine SchülerInnen mehr aufnehmen, die in Verbal-Reasoning weniger als 50,1 Punkte erreichen.
Bei gleicher Gewichtung beider Outputs könnte man es sich sogar leisten, Schüler mit Verbal-Reasoning-Scores von 48,6 aufzunehmen.

\subsection{Zusammenfassung}
Die Beurteilung von Bildungseinrichtungen mithife von DEA ist weit verbreitet.
In \cite{thanassoulis_guiding_1994} werden englische Schulen anhand von GCSE-Ergebnissen und dem Anteil der nicht-arbeitslosen AbsolventInnen verglichen, als Input-Parameter dienen Verbal-Reasoning-Testergebnisse und die von offizieller Seite bestätigte und mit Essensmarken bekämpfte soziale Benachteiligung (bzw. jener Anteil der SchülerInnen, die keine solche Hilfe in Anspruch nehmen).

Neben dem Vergleich zwischen den einzelnen Schulen identifiziert die Studie für ineffiziente Schulen "`Peers"', das sind Schulen mit ähnlichen Input-Parametern und besseren Outputs.
Dadurch können sich ineffiziente Schulen an besser performenden orientieren.

Die Studie macht außerdem deutlich, welch großer Einfluss von der Auswahl der herangezogenen Paramter ausgeht.
Da weder die Größe der Schule, noch die Anzahl der AbsolventInnen oder der Anteil der Dropouts vor dem GCSE-Test berücksichtigt wird, könnten Schulen ihre hier gemessene Effizienz verbessern, wenn schlechte SchülerInnen bereits vor dem Endtest ausscheiden und so gar nicht am GCSE-Examen teilnehmen.
% Tabelle 3.12